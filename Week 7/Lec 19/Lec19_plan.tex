%%%%%%%%%%%%%%%%%%%%%%%%%%%%%%%%%%%%
% Lesson Plan (50 minutes)
%%%%%%%%%%%%%%%%%%%%%%%%%%%%%%%%%%%%
\begin{frame}
    \frametitle{Lesson Plan}
    \begin{itemize}
        \item xx min Lecture: Test statistics, continued - motivate p-values
        \item xx min Lecture: p-values, one- vs two-sided tests
        \item xx min R Demonstration: p-values, significance level, tie in decision errors
        \item xx min Edfinity quiz: examples of interpreting and making decisions from p-values
        \item xx min Lecture: review quiz answers
    \end{itemize}
\end{frame}
            
%%%%%%%%%%%%%%%%%%%%%%%%%%%%%%%%%%%%
% Learning objectives:
%%%%%%%%%%%%%%%%%%%%%%%%%%%%%%%%%%%%
\begin{frame}
    \frametitle{Learning Objectives}
    \begin{itemize}
        \item \textbf{M1, LO3: Use R for Data Management and Exploration:} Utilize R to load, pre-process, and explore data through visualization and summarization techniques.
        \item \textbf{M3, LO4: Explain Hypothesis Testing and Its Limitations:} Discuss the use cases and potential issues with hypothesis testing, including the interpretation of results.
        \item \textbf{M3, LO5: Understand Errors and Significance Levels:} Identify Type I and Type II errors and explain how they are influenced by changes in the significance level.
        \item \textbf{M4, LO2: Design and Interpret Confidence Intervals:} Design, execute, and interpret confidence intervals for the population proportion.
        \item \textbf{M4, LO3: Conduct and Interpret Hypothesis Tests for Proportions:} Design, execute, and interpret hypothesis tests for population proportions.
    \end{itemize}
\end{frame}
    
%%%%%%%%%%%%%%%%%%%%%%%%%%%%%%%%%%%%
% TODO: Copy and adapt these slides base on the lesson plan
\input{.././Chp 5/5-3_ht_prop/5-3_ht_prop.tex}
    