%%%%%%%%%%%%%%%%%%%%%%%%%%%%%%%%%%%%
% Lesson Plan (50 minutes)
%%%%%%%%%%%%%%%%%%%%%%%%%%%%%%%%%%%%
\begin{frame}
    \frametitle{Lesson Plan}
    \begin{itemize}
        \item xx min Lecture: Motivate hypothesis tests
        \item xx min Lecture: Hypothesis testing framework
        \item xx min Lecture: Tying in confidence intervals
        \item xx min Lecture: Decision errors, error table
        \item xx min R Demonstration: examples of hypothesis tests and decision errors
        \begin{enumerate}
            \item Highlight: an error that might be easy to make in their own work
        \end{enumerate}
        \item xx min R Demonstration: computing error rates
        \item xx min Edfinity quiz: decision errors, HT concept check
        \item xx min Lecture: introduce test statistics (to be continued next time)
    \end{itemize}
\end{frame}
            
%%%%%%%%%%%%%%%%%%%%%%%%%%%%%%%%%%%%
% Learning objectives:
%%%%%%%%%%%%%%%%%%%%%%%%%%%%%%%%%%%%
\begin{frame}
    \frametitle{Learning Objectives}
    \begin{itemize}
        \item \textbf{M3, LO4: Explain Hypothesis Testing and Its Limitations:} Discuss the use cases and potential issues with hypothesis testing, including the interpretation of results.
        \item \textbf{M3, LO5: Understand Errors and Significance Levels:} Identify Type I and Type II errors and explain how they are influenced by changes in the significance level.
        \item \textbf{M4, LO2: Design and Interpret Confidence Intervals:} Design, execute, and interpret confidence intervals for the population proportion.
        \item \textbf{M4, LO3: Conduct and Interpret Hypothesis Tests for Proportions:} Design, execute, and interpret hypothesis tests for population proportions.
    \end{itemize}
\end{frame}
    
%%%%%%%%%%%%%%%%%%%%%%%%%%%%%%%%%%%%
% TODO: Copy and adapt these slides base on the lesson plan
\input{../../Chp 5/5-3_ht_prop/5-3_ht_prop.tex}
    