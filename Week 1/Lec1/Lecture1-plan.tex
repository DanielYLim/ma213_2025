%%%%%%%%%%%%%%%%%%%%%%%%%%%%%%%%%%%%
% Lesson Plan (50 minutes)
%%%%%%%%%%%%%%%%%%%%%%%%%%%%%%%%%%%%
\begin{frame}
    \frametitle{Lesson Plan}
    \begin{itemize}
        \item xx min Class discussion (?): getting to know the students: experience, interests, goals for the class, group formation
        \item xx min Lecture: course overview, syllabus
        \item xx min Lecture: data basics
        \item xx min Edfinity quiz: types of variables
    \end{itemize}
\end{frame}
    
%%%%%%%%%%%%%%%%%%%%%%%%%%%%%%%%%%%%
% Learning objectives:
%%%%%%%%%%%%%%%%%%%%%%%%%%%%%%%%%%%%
\begin{frame}
    \frametitle{Learning Objectives}
    \begin{itemize}
        \item \textbf{M1, LO1: Classify and Analyze Variables:} Categorize variables based on their types (e.g., numerical/categorical, continuous/discrete, ordinal), assess their association (positive, negative, or independent), and determine which make sense as explanatory vs. response variables.
        \item \textbf{M1, LO2: Evaluate Study Design and Its Implications:} Identify and explain experimental design choices (observational vs. experimental, sampling methods, blinding, potential biases), and judge whether results can be generalized to a population or used to infer causation. 
    \end{itemize}
\end{frame}

%%%%%%%%%%%%%%%%%%%%%%%%%%%%%%%%%%%%

% TODO: amend Lecture1-draft.tex. Depending on how much is covered, the rest of the material will be in Lecture 2.
% TODO: where to introduce R for the first time?
