%%%%%%%%%%%%%%%%%%%%%%%%%%%%%%%%%%%%
% Slide options
%%%%%%%%%%%%%%%%%%%%%%%%%%%%%%%%%%%%

% Option 1: Slides with solutions

\documentclass[slidestop,compress,mathserif]{beamer}
\newcommand{\soln}[1]{\textit{#1}}
\newcommand{\solnGr}[1]{#1}

% Option 2: Handouts without solutions

%\documentclass[11pt,containsverbatim,handout]{beamer}
%\usepackage{pgfpages}
%\pgfpagesuselayout{4 on 1}[letterpaper,landscape,border shrink=5mm]
%\newcommand{\soln}[1]{ }
%\newcommand{\solnGr}{ }


%%%%%%%%%%%%%%%%%%%%%%%%%%%%%%%%%%%%
% Style
%%%%%%%%%%%%%%%%%%%%%%%%%%%%%%%%%%%%

\def\chp3@path{../../Chp 3}
\input{../../lec_style.tex}


%%%%%%%%%%%%%%%%%%%%%%%%%%%%%%%%%%%%
% Preamble
%%%%%%%%%%%%%%%%%%%%%%%%%%%%%%%%%%%%

\title[Lecture 8]{MA213: Lecture 8}
\subtitle{Module 2: Probability, Random Variables, and Distributions}
\author{OpenIntro Statistics, 4th Edition}
\institute{$\:$ \\ {\footnotesize Based on slides developed by Mine \c{C}etinkaya-Rundel of OpenIntro. \\
The slides may be copied, edited, and/or shared via the \webLink{http://creativecommons.org/licenses/by-sa/3.0/us/}{CC BY-SA license.} \\
Some images may be included under fair use guidelines (educational purposes).}}
\date{}

%%%%%%%%%%%%%%%%%%%%%%%%%%%%%%%%%%%%
% Begin document
%%%%%%%%%%%%%%%%%%%%%%%%%%%%%%%%%%%%

\begin{document}


%%%%%%%%%%%%%%%%%%%%%%%%%%%%%%%%%%%%
% Title page
%%%%%%%%%%%%%%%%%%%%%%%%%%%%%%%%%%%%

{
\addtocounter{framenumber}{-1} 
{\removepagenumbers 
\usebackgroundtemplate{\includegraphics[width=\paperwidth]{../../OpenIntro_Grid_4_3-01.jpg}}
\begin{frame}

\hfill \includegraphics[width=20mm]{../../oiLogo_highres}

\titlepage

\end{frame}
}
}


%%%%%%%%%%%%%%%%%%%%%%%%%%%%%%%%%%%%
% Recap/Agenda 
%%%%%%%%%%%%%%%%%%%%%%%%%%%%%%%%%%%%
% TODO better formatting
\begin{frame}
    \frametitle{Module 2: Probability, Random Variables, and Distributions}
    \begin{itemize}
        \item \hl{Previously: } Conditional Probability, Sampling (Chapters 3.2-3.3)
        \item \hl{This time: } Random Variables (Chapter 3.4)
        \item \hl{Reading: } Chapter 3.5 for next time
        \item \hl{Deadlines/Announcements: } Quiz 1 in discussions next week
    \end{itemize}
    
\end{frame}
    

%%%%%%%%%%%%%%%%%%%%%%%%%%%%%%%%%%%%
% Sections
%%%%%%%%%%%%%%%%%%%%%%%%%%%%%%%%%%%%

%%%%%%%%%%%%%%%%%%%%%%%%%%%%%%%%%%%%

\section{Sampling from a small population}

%%%%%%%%%%%%%%%%%%%%%%%%%%%%%%%%%%%%

\begin{frame}
\frametitle{Sampling with replacement}

When sampling \hl{with replacement}, you put back what you just drew.

\pause

\begin{itemize}

\item Imagine you have a bag with 5 red, 3 blue and 2 orange chips in it. What is the probability that the first chip you draw is blue?
\begin{center}
5 \textcolor{red}{$\CIRCLE$}~, 3 \textcolor{blue}{$\CIRCLE$}~, 2 \textcolor{orange}{$\CIRCLE$}
\end{center}

\pause

\[ Prob(1^{st} \text{ chip } B) = \frac{3}{5 + 3 + 2} = \frac{3}{10} = 0.3 \]

\pause

\item Suppose you did indeed pull a blue chip in the first draw. If drawing with replacement, what is the probability of drawing a blue chip in the second draw?

\pause

\begin{center}
$1^{st}$ draw: 5 \textcolor{red}{$\CIRCLE$}~, 3 \textcolor{blue}{$\CIRCLE$}~, 2 \textcolor{orange}{$\CIRCLE$} \\

\pause

$2^{nd}$ draw: 5 \textcolor{red}{$\CIRCLE$}~, 3 \textcolor{blue}{$\CIRCLE$}~, 2 \textcolor{orange}{$\CIRCLE$}
\end{center}

\pause

\[ Prob(2^{nd} \text{ chip } B | 1^{st} \text{ chip } B) = \frac{3}{10} = 0.3 \]

\end{itemize}

\end{frame}

%%%%%%%%%%%%%%%%%%%%%%%%%%%%%%%%%%%%%

\begin{frame}
\frametitle{Sampling with replacement (cont.)}

\begin{itemize}

\item Suppose you actually pulled an orange chip in the first draw. If drawing with replacement, what is the probability of drawing a blue chip in the second draw?

\pause

\begin{center}
$1^{st}$ draw: 5 \textcolor{red}{$\CIRCLE$}~, 3 \textcolor{blue}{$\CIRCLE$}~, 2 \textcolor{orange}{$\CIRCLE$} \\
\pause
$2^{nd}$ draw: 5 \textcolor{red}{$\CIRCLE$}~, 3 \textcolor{blue}{$\CIRCLE$}~, 2 \textcolor{orange}{$\CIRCLE$}
\end{center}
\pause
\[ Prob(2^{nd} \text{ chip } B | 1^{st} \text{ chip } O) = \frac{3}{10} = 0.3 \]

\pause
\item If drawing with replacement, what is the probability of drawing two blue chips in a row?
\begin{center}

\pause
$1^{st}$ draw: 5 \textcolor{red}{$\CIRCLE$}~, 3 \textcolor{blue}{$\CIRCLE$}~, 2 \textcolor{orange}{$\CIRCLE$} \\
$2^{nd}$ draw: 5 \textcolor{red}{$\CIRCLE$}~, 3 \textcolor{blue}{$\CIRCLE$}~, 2 \textcolor{orange}{$\CIRCLE$}
\end{center}
\pause
\[ Prob(1^{st} \text{ chip } B) \cdot Prob(2^{nd} \text{ chip } B | 1^{st} \text{ chip } B) = 0.3 \times 0.3 \]
\[ = 0.3^2 = 0.09 \]

\end{itemize}

\end{frame}

%%%%%%%%%%%%%%%%%%%%%%%%%%%%%%%%%%%%%

\begin{frame}
\frametitle{Sampling with replacement (cont.)}

\begin{itemize}

\item When drawing with replacement, probability of the second chip being blue does not depend on the color of the first chip since whatever we draw in the first draw gets put back in the bag.
\[ Prob(B | B) = Prob(B | O) \]

\item In addition, this probability is equal to the probability of drawing a blue chip in the first draw, since the composition of the bag never changes when sampling with replacement.
\[ Prob(B | B) = Prob(B) \]

\item \hl{When drawing with replacement, draws are independent.}

\end{itemize}

\end{frame}


%%%%%%%%%%%%%%%%%%%%%%%%%%%%%%%%%%%%%

\begin{frame}
\frametitle{Sampling without replacement}

When drawing \hl{without replacement} you do not put back what you just drew.

\begin{itemize}

\pause

\item Suppose you pulled a blue chip in the first draw. If drawing without replacement, what is the probability of drawing a blue chip in the second draw?
\pause
\begin{center}
$1^{st}$ draw: 5 \textcolor{red}{$\CIRCLE$}~, 3 \textcolor{blue}{$\CIRCLE$}~, 2 \textcolor{orange}{$\CIRCLE$} \\
\pause
$2^{nd}$ draw: 5 \textcolor{red}{$\CIRCLE$}~, 2 \textcolor{blue}{$\CIRCLE$}~, 2 \textcolor{orange}{$\CIRCLE$}
\end{center}
\pause
\[ Prob(2^{nd} \text{ chip } B | 1^{st} \text{ chip } B) = \frac{2}{9} = 0.22 \]

\pause

\item If drawing without replacement, what is the probability of drawing two blue chips in a row?
\begin{center}

\pause
$1^{st}$ draw: 5 \textcolor{red}{$\CIRCLE$}~, 3 \textcolor{blue}{$\CIRCLE$}~, 2 \textcolor{orange}{$\CIRCLE$} \\
$2^{nd}$ draw: 5 \textcolor{red}{$\CIRCLE$}~, 2 \textcolor{blue}{$\CIRCLE$}~, 2 \textcolor{orange}{$\CIRCLE$}
\end{center}
\pause
\[ Prob(1^{st} \text{ chip } B) \cdot Prob(2^{nd} \text{ chip } B | 1^{st} \text{ chip } B)  = 0.3 \times 0.22 \]
\[ = 0.066 \]

\end{itemize}

\end{frame}

%%%%%%%%%%%%%%%%%%%%%%%%%%%%%%%%%%%%%

\begin{frame}
\frametitle{Sampling without replacement (cont.)}

\begin{itemize}

\item When drawing without replacement, the probability of the second chip being blue given the first was blue is not equal to the probability of drawing a blue chip in the first draw since the composition of the bag changes with the outcome of the first draw.
\[ Prob(B | B) \ne Prob(B) \]

\pause

\item \hl{When drawing without replacement, draws are not independent.}

\pause

\item This is especially important to take note of when the sample sizes are small. If we were dealing with, say, 10,000 chips in a (giant) bag, taking out one chip of any color would not have as big an impact on the probabilities in the second draw.

\end{itemize}

\end{frame}

%%%%%%%%%%%%%%%%%%%%%%%%%%%%%%%%%%%%%

% \begin{frame}
% \frametitle{Practice}

% \pq{In most card games cards are dealt without replacement. What is the probability of being dealt an ace and then a 3? Choose the closest answer.}

% \twocol{0.3}{0.6}{
% \begin{enumerate}[(a)]
% \item 0.0045
% \item 0.0059
% \solnMult{0.0060}
% \item 0.1553
% \end{enumerate}
% }
% {
% \soln{
% \pause
% \[ P(ace~then~3) = \frac{4}{52} \times \frac{4}{51} \approx 0.0060 \]
% }}

% \end{frame}


%%%%%%%%%%%%%%%%%%%%%%%%%%%%%%%%%%%%

\section{Edfinity Quiz}

%%%%%%%%%%%%%%%%%%%%%%%%%%%%%%%%%%%%

% \section{R Demonstration: Sampling}

%%%%%%%%%%%%%%%%%%%%%%%%%%%%%%%%%%%%

\section{Random variables}

%%%%%%%%%%%%%%%%%%%%%%%%%%%%%%%%%%%%

\begin{frame}
\frametitle{Random variables}

\begin{itemize}

\item A \hl{random variable} is a numeric quantity whose value depends on the outcome of a random event
\begin{itemize}
\item We use a capital letter, like $X$, to denote a random variable
\item The values of a random variable are denoted with a lowercase letter, in this case $x$
\item For example, $P(X = x)$
\end{itemize}

\item There are two types of random variables:
\begin{itemize}
\item \hl{Discrete random variables} often take only integer values
\begin{itemize}
\item Example: Number of credit hours, Difference in number of credit hours this term vs last
\end{itemize}
\item \hl{Continuous random variables} take real (decimal) values
\begin{itemize}
\item Example: Cost of books this term, Difference in cost of books this term vs last
\end{itemize}
\end{itemize}

\end{itemize}

\end{frame}

%%%%%%%%%%%%%%%%%%%%%%%%%%%%%%%%%%%%

\begin{frame}
\frametitle{Distribution of a discrete random variable}

\dq{In a game of cards you win \$1 if you draw a heart, \$5 if you draw an ace (including the ace of hearts), \$10 if you draw the king of spades and nothing for any other card you draw. Write the probability model for your winnings, and calculate your expected winning.}

\pause

\begin{center}
\renewcommand{\arraystretch}{1.5}
\begin{tabular}{l | c | c | c }
Event		& $X$ 		& $P(X)$        		 \\
\hline
Heart (not ace)	& $1$		& $\frac{12}{52}$	 \\
Ace			& $5$		& $\frac{4}{52}$	 \\	
King of spades	& $10$		& $\frac{1}{52}$	 \\	
All else		& $0$		& $\frac{35}{52}$	\\
\hline
Total			&			&		1		
\end{tabular}

\end{center}

\end{frame}

%%%%%%%%%%%%%%%%%%%%%%%%%%%%%%%%%%%%

\begin{frame}
\frametitle{Distribution of a discrete random variable}

Below is a visual representation of the probability distribution of winnings from this game:

\begin{center}
\includegraphics[width=0.8\textwidth]{\chp3@path/3-4_random_variables/figures/card_game/card_game}
\end{center}

\end{frame}

%%%%%%%%%%%%%%%%%%%%%%%%%%%%%%%%%%%%

%%%%%%%%%%%%%%%%%%%%%%%%%%%%%%%%%%%%

\section{Edfinity Quiz}

%%%%%%%%%%%%%%%%%%%%%%%%%%%%%%%%%%%%

%%%%%%%%%%%%%%%%%%%%%%%%%%%%%%%%%%%%

\subsection{Expectation}

%%%%%%%%%%%%%%%%%%%%%%%%%%%%%%%%%%%%

\begin{frame}
\frametitle{Expectation}

\begin{itemize}

\item We are often interested in the average outcome of a random variable.

\item We call this the \hl{expected value} (mean), and it is a weighted average of the possible outcomes
\formula{\[\mu = E(X) = \sum_{i = 1}^k x_i ~ P(X = x_i)\]}

\end{itemize}

\end{frame}

%%%%%%%%%%%%%%%%%%%%%%%%%%%%%%%%%%%%

\begin{frame}
\frametitle{Expected value of a discrete random variable}

\dq{In a game of cards you win \$1 if you draw a heart, \$5 if you draw an ace (including the ace of hearts), \$10 if you draw the king of spades and nothing for any other card you draw. Write the probability model for your winnings, and calculate your expected winning.}

\begin{center}
\renewcommand{\arraystretch}{1.5}
\begin{tabular}{l | c | c | c }
Event		& $X$ 		& $P(X)$        		& $X ~ P(X)$ \\
\hline
Heart (not ace)	& $1$		& $\frac{12}{52}$	& $\frac{12}{52}$ \\
Ace			& $5$		& $\frac{4}{52}$	& $\frac{20}{52}$ \\	
King of spades	& $10$		& $\frac{1}{52}$	& $\frac{10}{52}$ \\	
All else		& $0$		& $\frac{35}{52}$	& $0$ \\
\hline
Total			&			&		1		& $E(X) = \frac{42}{52} \approx 0.81$
\end{tabular}

\end{center}

\end{frame}

%%%%%%%%%%%%%%%%%%%%%%%%%%%%%%%%%%%%

\subsection{Variability in random variables}

%%%%%%%%%%%%%%%%%%%%%%%%%%%%%%%%%%%%

\begin{frame}
\frametitle{Variability}

We are also often interested in the variability in the values of a random variable.

\formula{
\[ \sigma^2 = Var(X) = \sum_{i = 1}^k (x_i - E(X))^2 P(X = x_i) \]
\[ \sigma = SD(X) = \sqrt{Var(X)} \]
}

\end{frame}

%%%%%%%%%%%%%%%%%%%%%%%%%%%%%%%%%%%%

\begin{frame}
\frametitle{Variability of a discrete random variable}

\dq{For the previous card game example, how much would you expect the winnings to vary from game to game?}

\vspace{2mm}
\only<2->{
{\footnotesize
\begin{center}
\renewcommand{\arraystretch}{2}
\begin{tabular}{c | c | c | l | p{4cm}}
$X$ & $P(X)$         & $X ~ P(X)$      & \multicolumn{1}{c|}{$(X - E(X))^2$}  & \multicolumn{1}{c}{$P(X) ~ (X - E(X))^2$}  \\
\hline
1 & $\frac{12}{52}$  & $1 \times \frac{12}{52} = \frac{12}{52}$ & $(1 - 0.81)^2 = 0.0361$ &  $\frac{12}{52} \times 0.0361 = 0.0083$ \\
\hline
5 & $\frac{4}{52}$   & $5 \times \frac{4}{52} = \frac{20}{52}$ & $(5 - 0.81)^2 = 17.5561$  & $\frac{4}{52} \times 17.5561 = 1.3505$ \\
\hline
10  & $\frac{1}{52}$ & $10 \times \frac{1}{52} = \frac{10}{52}$  & $(10 - 0.81)^2 = 84.4561$   & $\frac{1}{52} \times 84.0889 = 1.6242$ \\
\hline
0 & $\frac{35}{52}$  & $0 \times \frac{35}{52} = 0$  & $(0 - 0.81)^2 = 0.6561$ & $\frac{35}{52} \times 0.6561 = 0.4416$ \\
\hline
  &       & $E(X) = 0.81$ & & \soln{\only<3->{$V(X) = 3.4246$}} \\
 &       &                                                         & & \soln{\only<4>{$SD(X) = \sqrt{3.4246} = 1.85$}} \\
\end{tabular}
\end{center}
}
}
\end{frame}

%%%%%%%%%%%%%%%%%%%%%%%%%%%%%%%%%%%%

\subsection{Linear combinations of random variables}

%%%%%%%%%%%%%%%%%%%%%%%%%%%%%%%%%%%%

\begin{frame}
\frametitle{Linear combinations}

\begin{itemize}

\item A \hl{linear combination} of random variables $X$ and $Y$ is given by

\[ aX + bY \]

where $a$ and $b$ are some fixed numbers.

\pause

\item The average value of a linear combination of random variables is given by
\formula{\[ E(aX + bY) = a \times E(X) + b \times E(Y) \]}

\end{itemize}

\end{frame}

%%%%%%%%%%%%%%%%%%%%%%%%%%%%%%%%%%%%

\begin{frame}
\frametitle{Calculating the expectation of a linear combination}

(TODO: E[aX+bY])

% \dq{On average you take 10 minutes for each statistics homework problem and 15 minutes for each chemistry homework problem. This week you have 5 statistics and 4 chemistry homework problems assigned. What is the total time you expect to spend on statistics and chemistry homework for the week?}

% \soln{
% \pause
% \begin{align*} 
% E(S + S + S + S + S + C + C + C + C) &= 5 \times E(S) + 4 \times E(C) \\
% &= 5 \times 10 + 4 \times 15 \\
% &= 50 + 60 \\
% &= 110~min 
% \end{align*}
% }

\end{frame}

%%%%%%%%%%%%%%%%%%%%%%%%%%%%%%%%%%%%

\subsection{Variability in linear combinations of random variables}

%%%%%%%%%%%%%%%%%%%%%%%%%%%%%%%%%%%%

\begin{frame}
\frametitle{Linear combinations}

\begin{itemize}

\item The variability of a linear combination of two independent random variables is calculated as
\formula{\[ V(aX + bY) = a^2 \times V(X) + b^2 \times V(Y) \]}

\pause 

\item The standard deviation of the linear combination is the square root of the variance.

\end{itemize}

\pause 
\vfill

\Note{If the random variables are not independent, the variance calculation gets a little more complicated and is beyond the scope of this course.}

\end{frame}

%%%%%%%%%%%%%%%%%%%%%%%%%%%%%%%%%%%%

\begin{frame}
\frametitle{Calculating the variance of a linear combination}

(TODO: Var[aX+bY])

% \dq{The standard deviation of the time you take for each statistics homework problem is 1.5 minutes, and it is 2 minutes for each chemistry problem. What is the standard deviation of the time you expect to spend on statistics and physics homework for the week if you have 5 statistics and 4 chemistry homework problems assigned? Suppose that the time it takes to complete each problem is independent of another.}

% \soln{
% \pause
% \begin{align*} 
% V(S + S + S + S + S + C + C + C + C) &= V(S) + V(S) + V(S) + V(S) + V(S) + V(C) + V(C) + V(C) + V(C) \\
% &= 5 \times V(S) + 4 \times V(C) \\
% &= 5 \times 1.5^2 + 4 \times 2^2 \\
% &= 27.25
% \end{align*}
% }

\end{frame}

%%%%%%%%%%%%%%%%%%%%%%%%%%%%%%%%%%%%

\section{Edfinity Quiz}

%%%%%%%%%%%%%%%%%%%%%%%%%%%%%%%%%%%%



%%%%%%%%%%%%%%%%%%%%%%%%%%%%%%%%%%%%
% End document
%%%%%%%%%%%%%%%%%%%%%%%%%%%%%%%%%%%%

\end{document}