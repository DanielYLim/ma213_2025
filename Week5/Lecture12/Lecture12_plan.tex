%%%%%%%%%%%%%%%%%%%%%%%%%%%%%%%%%%%%
% Lesson Plan (50 minutes)
%%%%%%%%%%%%%%%%%%%%%%%%%%%%%%%%%%%%
\begin{frame}
    \frametitle{Lesson Plan}
    \begin{itemize}
        \item xx min Lecture: Bernoulli distribution, Bernoulli random variables (motivate Geometric)
        \item xx min Edfinity quiz: Bernoulli random variables
        \item xx min Board work: Derivation of Geometric distribution from Bernoulli
        \item xx min Lecture: properties of the Geometric distribution
        \begin{itemize}
            \item Probability mass function
            \item Formulas for expectation and variance
            \item Variability
        \end{itemize}
        \item xx min R Demonstration: sampling from Geometric, different parameters
        \item xx min Edfinity quiz: variability of Geometric random variables
        \item xx min Lecture: review quiz answers
        \item Next time: Binomial (another Bernoulli-linked distribution)
    \end{itemize}
\end{frame}

%%%%%%%%%%%%%%%%%%%%%%%%%%%%%%%%%%%%
% Learning objectives:
%%%%%%%%%%%%%%%%%%%%%%%%%%%%%%%%%%%%
\begin{frame}
    \frametitle{Learning Objectives}
    \begin{itemize}
        \item \textbf{M1 LO3: Use R for Data Management and Exploration:} Utilize R to load, pre-process, and explore data through visualization and summarization techniques.
        \item \textbf{M2 LO1: Validate and Explain Probability Distributions:} Assess the validity of a probability distribution using the concepts of outcome, sample space, and probability properties (e.g., disjoint outcomes, probabilities between 0 and 1, and total probabilities summing to 1).
        \item \textbf{M2 LO4: Understand and Compute Expectations and Variances:} Explain the concepts of expectations and variances of random variables, and compute the expectation and variance of a linear combination of random variables.
        \item \textbf{M2 LO5: Model Data Using Bernoulli, Geometric, and Binomial Distributions:} Recognize when to appropriately model data using the Bernoulli, geometric, and binomial distributions, and compute quantities of interest such as mean, standard deviation, and tail probabilities.
    \end{itemize}
\end{frame}

%%%%%%%%%%%%%%%%%%%%%%%%%%%%%%%%%%%%
% TODO: Adapt drafted slides

