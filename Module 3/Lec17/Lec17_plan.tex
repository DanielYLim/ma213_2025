

%%%%%%%%%%%%%%%%%%%%%%%%%%%%%%%%%%%%
% Lesson Plan (50 minutes)
%%%%%%%%%%%%%%%%%%%%%%%%%%%%%%%%%%%%
\begin{frame}
\frametitle{Lesson Plan}
\begin{itemize}
    \item xx min Lecture: Motivate CIs
    \item xx min Board work: given N(mu, sterr), derivation of interval with target level
    \item xx min Lecture: Facebook example
    \item xx min R demonstration: simulate running the experiment many times, get different answers for the CI, 95\% contain true
    \item xx min Edfinity quiz (confidence level)
    \item xx min R demonstration: effect of confidence level
    \item xx min lecture: Wrap up confidence intervals
\end{itemize}
\end{frame}
        
%%%%%%%%%%%%%%%%%%%%%%%%%%%%%%%%%%%%
% Learning objectives:
%%%%%%%%%%%%%%%%%%%%%%%%%%%%%%%%%%%%
\begin{frame}
\frametitle{Learning Objectives}
\begin{itemize}
    \item Calculate and Interpret Standard Error: Calculate the standard error for proportions and interpret it as a measure of sampling variability. [Q3, L4]
    \item (preview Module 4) Design and Interpret Confidence Intervals: Design, execute, and interpret confidence intervals for the population proportion. [Q4, L5] Core 
\end{itemize}
\end{frame}

%%%%%%%%%%%%%%%%%%%%%%%%%%%%%%%%%%%%
% TODO: Copy and adapt these slides base on the lesson plan
\input{\chp5@path/5-2_ci_prop/5-2_ci_prop}
