%%%%%%%%%%%%%%%%%%%%%%%%%%%%%%%%%%%%
% Lesson Plan (50 minutes)
%%%%%%%%%%%%%%%%%%%%%%%%%%%%%%%%%%%%
\begin{frame}
    \frametitle{Lesson Plan}
    \begin{itemize}
        \item xx min Lecture: recap pooled proportion
        \item xx min R Demonstration: examples comparing one/two proportion inference computations
        \item xx min Lecture: new topic, introduce/motivate chi-squared tests
        \item xx min Edfinity quiz: basically slides 31-32 (practice setting hypotheses)
        \item xx min Lecture/board work: derive(?) chi-square statistic, chi-squared distribution
        \item xx min R Demonstration: show and interpret some examples of chi-squared tests of GOF
        \item xx min Edfinity quiz: putting chi-squared GOF tests into practice (like slides 47, 53-54)
        % I like the idea of having a motivating example (Labby's dice here) to return to throughout the lecture,
        % if the dice are too simplistic/boring can think up another one. if this sounds good, one last item
        % I would add to close out this lecture would be like slide 44 (back to the example)
    \end{itemize}
\end{frame}
            
%%%%%%%%%%%%%%%%%%%%%%%%%%%%%%%%%%%%
% Learning objectives:
%%%%%%%%%%%%%%%%%%%%%%%%%%%%%%%%%%%%
\begin{frame}
    \frametitle{Learning Objectives}
    \begin{itemize}
        \item \textbf{M1, LO3: Use R for Data Management and Exploration:} Utilize R to load, pre-process, and explore data through visualization and summarization techniques.
        \item \textbf{M3, LO3: Calculate and Interpret Standard Error:} Calculate the standard error for proportions and interpret it as a measure of sampling variability.
        \item \textbf{M4, LO2: Design and Interpret Confidence Intervals:} Design, execute, and interpret confidence intervals for the population proportion.
        \item \textbf{M4, LO3: Conduct and Interpret Hypothesis Tests for Proportions:} Design, execute, and interpret hypothesis tests for population proportions.
        \item \textbf{M4, LO4: Conduct and Interpret Chi-Square Tests:} Assess whether the conditions for a chi-square test (goodness of fit or independence) are met, and if so, design, execute, and interpret the test.
    \end{itemize}
\end{frame}
    
%%%%%%%%%%%%%%%%%%%%%%%%%%%%%%%%%%%%
% TODO: Copy and adapt these slides base on the lesson plan
<<<<<<< HEAD
<<<<<<< HEAD
\input{../Chp 6/6-3_chisq_gof/6-3_chisq_gof.tex}
=======
\input{../Chp 6/6-3_chisq_gof/6-3_chisq_gof.tex}
>>>>>>> Fixed typo
=======
\input{../../Chp 6/6-3_chisq_gof/6-3_chisq_gof.tex}
>>>>>>> Updated paths in input commands
