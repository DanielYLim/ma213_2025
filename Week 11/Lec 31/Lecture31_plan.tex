%%%%%%%%%%%%%%%%%%%%%%%%%%%%%%%%%%%%
% Lesson Plan (50 minutes)
%%%%%%%%%%%%%%%%%%%%%%%%%%%%%%%%%%%%
\begin{frame}
    \frametitle{Lesson Plan}
    \begin{itemize}
        \item xx min Lecture: review residuals, correlation from last time
        \item xx min Lecture: how to "find" the best-fit line? different objectives - motivate least squares (like slide 12)
        \item xx min Lecture: conditions for least squares (linearity, nearly-normal residuals, constant variability)
        \item xx min R Demonstration: for an example problem, determine whether the conditions for least squares have been met 
        \item xx min Edfinity quiz: new examples, are the conditions met?
        \item xx min Lecture: slope and intercept, and their interpretations in regression 
        \item xx min R Demonstration: slope and intercept in the lecture example (from scratch, and using lm)
        \item xx min Edfinity quiz: practice interpreting the slope and intercept in new examples
        \item xx min R Demonstration: now that we have the slope and intercept, we can plot a regression line and make predictions
        \item xx min R Demonstration: extrapolation
        \item (if time) xx min Lecture: $R^2$ and its interpretation (can continue later)
    \end{itemize}
\end{frame}

%%%%%%%%%%%%%%%%%%%%%%%%%%%%%%%%%%%%
% Learning objectives:
%%%%%%%%%%%%%%%%%%%%%%%%%%%%%%%%%%%%
\begin{frame}
    \frametitle{Learning Objectives}
    \begin{itemize}
        \item \textbf{M1, LO1: Classify and Analyze Variables:} Categorize variables based on their types (e.g., numerical/categorical, continuous/discrete, ordinal), assess their association (positive, negative, or independent), and determine which make sense as explanatory vs. response variables.
        \item \textbf{M1, LO3: Use R for Data Management and Exploration:} Utilize R to load, pre-process, and explore data through visualization and summarization techniques.
        \item \textbf{M1, LO4: Visualize and Describe Data Distributions:} Select appropriate visualizations (scatterplots, histograms, box plots, bar plots) to depict data, and describe distributions qualitatively (shape, center, spread, outliers) and quantitatively (mean, median, mode, range, IQR, standard deviation).
        \item \textbf{M5, LO1: Describe and Assess Relationships Between Two Variables:} Describe the association between two numerical variables in a scatter plot in terms of direction, shape (linear or nonlinear), and strength, and assess whether linear regression is an appropriate model.    
        \item \textbf{M5, LO2: Compute and Interpret Correlation and R²:} Compute and interpret correlation coefficients and R² values, while recognizing that correlation does not imply causation. 
        \item \textbf{M5, LO3: Fit and Interpret Linear Models Using Least Squares:} Fit the intercept and slope of a linear model to data using the least squares method, interpret the fitted values, and use the model to predict responses to new inputs.
    \end{itemize}
\end{frame}
    
%%%%%%%%%%%%%%%%%%%%%%%%%%%%%%%%%%%%
% TODO: Copy and adapt these slides base on the lesson plan
\input{../../Chp 8/8-2_least_square_reg/8-2_least_square_reg.tex}