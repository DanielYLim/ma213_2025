%%%%%%%%%%%%%%%%%%%%%%%%%%%%%%%%%%%%
% Slide options
%%%%%%%%%%%%%%%%%%%%%%%%%%%%%%%%%%%%

% Option 1: Slides with solutions

\documentclass[slidestop,compress,mathserif]{beamer}
\newcommand{\soln}[1]{\textit{#1}}
\newcommand{\solnGr}[1]{#1}

% Option 2: Handouts without solutions

%\documentclass[11pt,containsverbatim,handout]{beamer}
%\usepackage{pgfpages}
%\pgfpagesuselayout{4 on 1}[letterpaper,landscape,border shrink=5mm]
%\newcommand{\soln}[1]{ }
%\newcommand{\solnGr}{ }

%%%%%%%%%%%%%%%%%%%%%%%%%%%%%%%%%%%%
% Style
%%%%%%%%%%%%%%%%%%%%%%%%%%%%%%%%%%%%

\input{../../lec_style.tex}
 

%%%%%%%%%%%%%%%%%%%%%%%%%%%%%%%%%%%%
% Preamble
%%%%%%%%%%%%%%%%%%%%%%%%%%%%%%%%%%%%

\title[Lecture 1]{MA213: Lecture 1}
\subtitle{Module 1: Exploratory Data Analysis and Study Design}
\author{OpenIntro Statistics, 4th Edition}
\institute{$\:$ \\ {\footnotesize Based on slides developed by Mine \c{C}etinkaya-Rundel of OpenIntro. \\
The slides may be copied, edited, and/or shared via the \webLink{http://creativecommons.org/licenses/by-sa/3.0/us/}{CC BY-SA license.} \\
Some images may be included under fair use guidelines (educational purposes).}}
\date{}

%%%%%%%%%%%%%%%%%%%%%%%%%%%%%%%%%%%%
% Begin document
%%%%%%%%%%%%%%%%%%%%%%%%%%%%%%%%%%%%

\begin{document}


%%%%%%%%%%%%%%%%%%%%%%%%%%%%%%%%%%%%
% Title page
%%%%%%%%%%%%%%%%%%%%%%%%%%%%%%%%%%%%

{
\addtocounter{framenumber}{-1} 
{\removepagenumbers 
\usebackgroundtemplate{\includegraphics[width=\paperwidth]{../../OpenIntro_Grid_4_3-01.jpg}}
\begin{frame}

\hfill \includegraphics[width=20mm]{../../oiLogo_highres}

\titlepage

\end{frame}
}
}

%%%%%%%%%%%%%%%%%%%%%%%%%%%%%%%%%%%%
% Sections
%%%%%%%%%%%%%%%%%%%%%%%%%%%%%%%%%%%%

%%%%%%%%%%%%%%%%%%%%%%%%%%%%%%%%%%%%
\section{Course Introduction}
%%%%%%%%%%%%%%%%%%%%%%%%%%%%%%%%%%%%

%%%%%%%%%%%%%%%%%%%%%%%%%%%%%%%%%%%
% Welcome
\begin{frame}
	\frametitle{Welcome to MA213: Basic Probability and Statistics}
	This is a newly redesigned course!
	\begin{itemize}
		\item \hl{We tried to design the course to be} accessible, engaging, and relevant.
		\item \hl{We want you to feel} connected to the course as a community and supported in your learning.
		\item \hl{We hope you will take away}
		\begin{itemize}
			\item A conceptual understanding of randomness and variability
			\item Practical skills for analyzing and interpreting data
			\item Confidence in your ability to work with data
			\item An appreciation for the beauty of statistics!
		\end{itemize}
		\item \hl{We will need your help} to make this course a success!
	\end{itemize}
\end{frame}

%%%%%%%%%%%%%%%%%%%%%%%%%%%%%%%%%%%
% Logistics
\begin{frame}
	\frametitle{Class Logistics}
	\begin{itemize}
		\item \hl{Lectures}: Mon, Wed, Fri 11:15am-12:05pm
		\item \hl{Labs}: Fridays
		\begin{itemize}
			\item Check your schedule for your time and location
			\item Work in groups, practice analyzing data in R
			\item Start this week!
		\end{itemize}
		\item \hl{Discussions}: Once a week
		\begin{itemize}
			\item Check your schedule for your time and location
			\item Quizzes, review, practice problems
			\item Start next week!
		\end{itemize}
	\end{itemize}
\end{frame}

%%%%%%%%%%%%%%%%%%%%%%%%%%%%%%%%%%%
% People
\begin{frame}
	\frametitle{People}
	\begin{itemize}
		\item \hl{Instructor:} Prof. Emily Stephen
		\item \hl{Labs:} Dr. Yongho Lim
		\item \hl{Graduate Teaching Fellows (TFs):} 
		\begin{itemize}
			\item 1
			\item 2
			\item 3
		\end{itemize}
		\item \hl{Undergraduate Learning Assistants (LAs):}
		\begin{itemize}
			\item Yao Lu
			\item Jack Hincks
		\end{itemize}
	\end{itemize}

	See the course website for office hours and contact info
\end{frame}

%%%%%%%%%%%%%%%%%%%%%%%%%%%%%%%%%%%
% Course Webpage

\begin{frame}
	\frametitle{Course Webpage} 
	\begin{itemize}
		\item \hl{Course website (Blackboard):} \url{https://learn.bu.edu} \\
		(Log in with your BU credentials)
		\item \hl{What's there:}
		\begin{itemize}
			\item Announcements
			\item Course documents: Syllabus (with Office Hours and Calendar), GenAI Policy, Learning Objectives
			\item Links to Textbook, Edfinity, Gradescope
			\item Course Forum
			\item Lecture slides and videos
			\item Lab materials
			\item Your gradebook
		\end{itemize}
	\end{itemize}	
\end{frame}

%%%%%%%%%%%%%%%%%%%%%%%%%%%%%%%%%%%
% Textbook

\begin{frame}
	\frametitle{Textbook: OpenIntro}
	\begin{columns}
		\column{0.6\textwidth}
		\begin{minipage}[t]{\linewidth}
		\vspace{0pt}
			\hl{OpenIntro Statistics}, 4th Edition \\
			David Diez, Mine \c{C}etinkaya-Rundel, and Christopher Barr \\
			\\
			\webLink{https://www.openintro.org/book/os/}{https://www.openintro.org/book/os/} \\
			\\
			Free download as PDF or \$25 for print copy
		\end{minipage}
		\column{0.4\textwidth}
		\begin{minipage}[t]{\linewidth}
		\vspace{0pt}
		\centering
		\includegraphics[width=0.9\columnwidth]{openintro_cover.jpeg}
		\end{minipage}
	\end{columns}
\end{frame}

%%%%%%%%%%%%%%%%%%%%%%%%%%%%%%%%%%%
% Learning Objectives

\begin{frame}
	\frametitle{Learning Objectives}

	\hl{Core Learning Objectives:} 19 core learning objectives \\
	\hl{Auxiliary Learning Objectives:} 11 auxiliary learning objectives \\

	\vspace{10pt}
	Module 1: Exploratory Data Analysis and Study Design \\(Chapters 1-2)
	\begin{enumerate}
		\item (core, Q1) Classify and analyze variables 
		\item (core, Q1) Evaluate Study Design and Its Implications 
		\item (core, Q1) Use R for Data Management and Exploration (core, Q1)
		\item (core, Q1) Visualize and Describe Data Distributions 
		\item (aux, Q1) Conduct Hypothesis Testing Using Simulation 
	\end{enumerate}
	See the syllabus for details
\end{frame}

\begin{frame}
	\frametitle{Learning Objectives}
	Module 6: Global module (core)
	\begin{enumerate}
		\item I can carry out a complete, reproducible statistical workflow in R using the inference methods from the course [Lab Projects] 
		\item Given R code for a statistical analysis, I can explain what it does and why, and identify both programmatic and statistical errors [Quizzes] 
		\item When solving probability and statistics problems, I can support my answers by writing out the steps using the notation and conventions of statistical exposition [Quizzes] 
		\item I can recognize whether a statistical workflow is appropriate for the given data and data analysis goals, and explain the results of a statistical analysis to stakeholders [Lab Projects]
	\end{enumerate}
\end{frame}

%%%%%%%%%%%%%%%%%%%%%%%%%%%%%%%%%%%
% Labs

\begin{frame}
	\frametitle{Labs} 
	Led by Yongho Lim, with support from TFs and LAs
	\begin{itemize}
		\item \hl{Weekly labs} on Fridays, starting this week
		\item Practice analyzing data and running simulations in R
		\item \hl{Skills labs}
		\begin{itemize}
			\item Work in pairs
			\item Analyze data and run simulations in R
			\item Post-lab questions to be submitted on Gradescope
		\end{itemize}
		\item \hl{Lab projects}
		\begin{itemize}
			\item Work in groups to explore real data
			\item Project 1: Data analysis video presentation
			\item Project 2: Statistical report
		\end{itemize}
	\end{itemize}
	More details in Lab!
\end{frame}

%%%%%%%%%%%%%%%%%%%%%%%%%%%%%%%%%%%
% Assignments and Grading Structure

\begin{frame}
	\frametitle{Assignments}
	\begin{itemize}
		\item 12 Weekly Edfinity \hl{homeworks}
		\begin{itemize}
			\item Due Mondays at 3:00pm
			\item Low-stakes, graded for completion
			\item Can revise and resubmit for full credit
			\item \hl{Homework 1} due Monday
		\end{itemize} 
		\item 5 \hl{Quizzes} 
		\begin{itemize}
			\item 19 Core Learning Objectives, 11 Auxiliary Learning Objectives
			\item 5-6 written problems, graded pass / almost pass / not yet
			\item In discussion sections, see calendar in Syllabus
			\item Can \textbf{qualify} to retake 3 quizzes at the end of the semester (details later)
		\end{itemize}
		\item 7 \hl{Skills Labs}
		\item 2 \hl{Lab Projects}
	\end{itemize}
\end{frame}

\begin{frame}
	\frametitle{Grading Structure}
	\begin{table}[ht]
	\centering
	\small
	\begin{tabular}{|l|c|c|c|}
	\hline
	 & \textbf{A} & \textbf{B} & \textbf{C} \\
	\hline
	Homeworks & 12/12 complete & 11/12 complete & 10/12 complete \\
	\hline
	Quizzes (Core) & 18/19 passed & 15/19 passed & 12/19 passed \\
	\hline
	Quizzes (Aux) & 9/11 passed & 6/11 passed & 0/11 passed \\
	\hline
	Skills Labs & 7/7 passed & Labs 1-6 passed & Labs 1-6 passed \\
	\hline
	Lab Projects & 2 satisfactory & 2 satisfactory & 2 satisfactory \\
	\hline
	Lectures & $>$34 attended & $>$28 attended & $>$20 attended \\
	\hline
	\end{tabular}
	\end{table}

	\hl{Additional factors:}
	\begin{itemize}
		\item Grades between letter grades will be determined by how close you are to the next letter grade
		\item Each ``unsatisfactory'' project will drop your course grade by half of a letter grade (e.g. B becomes B-)
	\end{itemize}
\end{frame}

%%%%%%%%%%%%%%%%%%%%%%%%%%%%%%%%%%%

\begin{frame}
	\frametitle{Class Policies} 
	\begin{itemize}
		\item \hl{Attendance and participation:} 
		\begin{itemize}
			\item Expected to attend all lectures, labs, and discussions. 
			\item In-class activities and participation will be part of your grade.
			\item Email the instructor or your TF if you need to miss class/discussion/lab
		\end{itemize}
		\item \hl{Academic integrity:} 
		\begin{itemize}
			\item All work must be your own. 
			\item Collaboration is allowed on homeworks and labs/projects, but \textbf{not on quizzes}. 
			\item See syllabus for details.
		\end{itemize}
		\item \hl{Use of Generative AI tools (e.g. ChatGPT):}
		\begin{itemize}
			\item Quizzes: \textbf{Not allowed}.
			\item Homeworks: Use at your own discretion
			\item Labs/Projects: Allowed for \textbf{specific uses} with \textbf{proper documentation}.
			\item See syllabus and GenAI policy document for details.
		\end{itemize}
	\end{itemize}
\end{frame}

%%%%%%%%%%%%%%%%%%%%%%%%%%%%%%%%%%%

%%%%%%%%%%%%%%%%%%%%%%%%%%%%%%%%%%%%
\section{A Case Study}
%%%%%%%%%%%%%%%%%%%%%%%%%%%%%%%%%%%%

% (Module 1 content) Case study TODO paste slides from OpenIntro

%%%%%%%%%%%%%%%%%%%%%%%%%%%%%%%%%%%
% Think/pair/share: Do the data show a "real" difference between groups? Are the results generalizable?
\begin{frame}
	\frametitle{Think/Pair/Share: Case Study}
	\begin{itemize}
		\item \textbf{Think:} Do the data show a "real" difference between groups? Are the results generalizable?
		\item \textbf{Pair:} Discuss with a partner.
		\item \textbf{Share:} Share your thoughts in the class discussion forum.
	\end{itemize}

	The class discussion board is under the \textit{Discussions} tab in the course website. % TODO
\end{frame}
%%%%%%%%%%%%%%%%%%%%%%%%%%%%%%%%%%%

% Big picture of statistics TODO

%%%%%%%%%%%%%%%%%%%%%%%%%%%%%%%%%%%%

% If time, start data basics slides  % TODO

%%%%%%%%%%%%%%%%%%%%%%%%%%%%%%%%%%%
\section{Edfinity Classroom Survey}
%%%%%%%%%%%%%%%%%%%%%%%%%%%%%%%%%%%

\begin{frame}
	\frametitle{How to access Edfinity} % TODO
\end{frame}

%%%%%%%%%%%%%%%%%%%%%%%%%%%%%%%%%%%

% Final announcements (Plan for next time, Reading, Edfinity survey, labs/discussions start week 2) TODO

%%%%%%%%%%%%%%%%%%%%%%%%%%%%%%%%%%%%
% Recap/Agenda 
%%%%%%%%%%%%%%%%%%%%%%%%%%%%%%%%%%%%
% TODO better formatting
\begin{frame}
    \frametitle{Lecture 1: Agenda}
    \begin{itemize}
        %\item \hl{Previously: }Probability distributions and random variables (Chapter 4)
        \item \hl{This time: }Course introduction
        \item \hl{Reading: }Chapter 1 for next time
        \item \hl{Deadlines/Announcements: }
    \end{itemize}
    
\end{frame}
    

%%%%%%%%%%%%%%%%%%%%%%%%%%%%%%%%%%%

%%%%%%%%%%%%%%%%%%%%%%%%%%%%%%%%%%%%
% End document
%%%%%%%%%%%%%%%%%%%%%%%%%%%%%%%%%%%%

\end{document}