%%%%%%%%%%%%%%%%%%%%%%%%%%%%%%%%%%%%
% Slide options
%%%%%%%%%%%%%%%%%%%%%%%%%%%%%%%%%%%%

% Option 1: Slides with solutions

\documentclass[slidestop,compress,mathserif]{beamer}
\newcommand{\soln}[1]{\textit{#1}}
\newcommand{\solnGr}[1]{#1}

% Option 2: Handouts without solutions

%\documentclass[11pt,containsverbatim,handout]{beamer}
%\usepackage{pgfpages}
%\pgfpagesuselayout{4 on 1}[letterpaper,landscape,border shrink=5mm]
%\newcommand{\soln}[1]{ }
%\newcommand{\solnGr}{ }

%%%%%%%%%%%%%%%%%%%%%%%%%%%%%%%%%%%%
% Style
%%%%%%%%%%%%%%%%%%%%%%%%%%%%%%%%%%%%
\def\chp7@path{../../Chp 7}
\input{../../lec_style.tex}


%%%%%%%%%%%%%%%%%%%%%%%%%%%%%%%%%%%%
% Preamble
%%%%%%%%%%%%%%%%%%%%%%%%%%%%%%%%%%%%

\title[Lecture 27]{MA213: Lecture 27}
\subtitle{Module 4: Inference}
\author{OpenIntro Statistics, 4th Edition}
\institute{$\:$ \\ {\footnotesize Based on slides developed by Mine \c{C}etinkaya-Rundel of OpenIntro. \\
The slides may be copied, edited, and/or shared via the \webLink{http://creativecommons.org/licenses/by-sa/3.0/us/}{CC BY-SA license.} \\
Some images may be included under fair use guidelines (educational purposes).}}
\date{}


%%%%%%%%%%%%%%%%%%%%%%%%%%%%%%%%%%%%
% Begin document
%%%%%%%%%%%%%%%%%%%%%%%%%%%%%%%%%%%%

\begin{document}


%%%%%%%%%%%%%%%%%%%%%%%%%%%%%%%%%%%%
% Title page
%%%%%%%%%%%%%%%%%%%%%%%%%%%%%%%%%%%%

{
\addtocounter{framenumber}{-1} 
{\removepagenumbers 
\usebackgroundtemplate{\includegraphics[width=\paperwidth]{../../OpenIntro_Grid_4_3-01.jpg}}
\begin{frame}

\hfill \includegraphics[width=20mm]{../../oiLogo_highres}

\titlepage

\end{frame}
}
}


%%%%%%%%%%%%%%%%%%%%%%%%%%%%%%%%%%%%
% Sections
%%%%%%%%%%%%%%%%%%%%%%%%%%%%%%%%%%%%

%%%%%%%%%%%%%%%%%%%%%%%%%%%%%%%%%%%%

\section{Motivation}
% Note: spitballed the language below, please feel free to formalize it, or scrap entirely
\begin{frame}
    \frametitle{Measuring the ``strength'' of our inferences}
    Last week, we did a lot of tests, computed confidence intervals and p-values, and made inferences. It would be natural to ask how ``strong" our conclusions are. 

    Is there another way to measure the ``strength" of these conclusions, in addition to test significance levels?
\end{frame}

% Introduce motivating example here?

%%%%%%%%%%%%%%%%%%%%%%%%%%%%%%%%%%%%

% Note: might want to rename this section something like "review of decision errors" and then make a new section for computing power
\section{Computing the power for a 2-sample test}

%%%%%%%%%%%%%%%%%%%%%%%%%%%%%%%%%%%%

\begin{frame}
\frametitle{}

\begin{center}
\begin{tabular}{l l | c c}
\multicolumn{2}{c}{} & \multicolumn{2}{c}{\textbf{Decision}} \\
& & fail to reject $H_0$ &  reject $H_0$ \\
  \cline{2-4}
& $H_0$ true & \onslide<4->{\green{$1 - \alpha$}} & \onslide<2->{\orange{Type 1 Error, $\alpha$}} \\
\raisebox{1.5ex}{\textbf{Truth}} & $H_A$ true &  \onslide<3->{\orange{Type 2 Error, $\beta$}} & \onslide<5->{\green{Power, $1 - \beta$}} \\
  \cline{2-4}
\end{tabular}
\end{center}

\pause

\begin{itemize}
\item Type 1 error is rejecting $H_0$ when you shouldn't have, and the probability of doing so is $\alpha$ (significance level)

\pause 

\item Type 2 error is failing to reject $H_0$ when you should have, and the probability of doing so is $\beta$ (a little more complicated to calculate)

\pause 

\item \hl{Power} of a test is the probability of correctly rejecting $H_0$, and the probability of doing so is $1 - \beta$

\pause 

\item In hypothesis testing, we want to keep $\alpha$ and $\beta$ low, but there are inherent trade-offs.

\end{itemize}

\end{frame}

%%%%%%%%%%%%%%%%%%%%%%%%%%%%%%%%%%%%

\begin{frame}
\frametitle{Type 2 error rate}

If the alternative hypothesis is actually true, what is the chance that we make a Type 2 Error, i.e. we fail to reject the null hypothesis even when we should reject it?

\begin{itemize}

\item The answer is not obvious.

\item If the true population average is very close to the null hypothesis value, it will be difficult to detect a difference (and reject $H_0$).

\item If the true population average is very different from the null hypothesis value, it will be easier to detect a difference.

\item Clearly, $\beta$ depends on the \hl{effect size} ($\delta$)
\end{itemize}

\end{frame}

%%%%%%%%%%%%%%%%%%%%%%%%%%%%%%%%%%%%

\section{Edfinity Quiz: Decision errors (review)}

%%%%%%%%%%%%%%%%%%%%%%%%%%%%%%%%%%%%

\begin{frame}
\frametitle{Example - Blood Pressure (BP), hypotheses}

{\dq
{\footnotesize
Suppose a pharmaceutical company has developed a new drug for lowering blood pressure, and they are preparing a clinical trial to test the drug's effectiveness. They recruit people who are taking a particular standard blood pressure medication, and half of the subjects are given the new drug (treatment) and the other half continue to take their current medication through generic-looking pills to ensure blinding (control). What are the hypotheses for a two-sided hypothesis test in this context?
}
}

\pause

\soln{
\begin{align*}
H_0&: \mu_{treatment} - \mu_{control} = 0 \\
H_A&: \mu_{treatment} - \mu_{control} \ne 0  
\end{align*}
}

\end{frame}

%%%%%%%%%%%%%%%%%%%%%%%%%%%%%%%%%%%%

\begin{frame}
\frametitle{Example - BP, standard error}

{\dq
{\footnotesize
Suppose researchers would like to run the clinical trial on patients with systolic blood pressures between 140 and 180 mmHg. Suppose previously published studies suggest that the standard deviation of the patients' blood pressures will be about 12 mmHg and the distribution of patient blood pressures will be approximately symmetric. If we had 100 patients per group, what would be the approximate standard error for difference in sample means of the treatment and control groups?
}
}

\pause

\soln{
\[ SE = \sqrt{ \frac{12^2}{100} + \frac{12^2}{100} } = 1.70 \]
}

\end{frame}

%%%%%%%%%%%%%%%%%%%%%%%%%%%%%%%%%%%%

\begin{frame}
\frametitle{Example - BP, minimum effect size required to reject $H_0$}

{\dq
{\footnotesize
For what values of the difference between the observed averages of blood pressure in treatment and control groups (effect size) would we reject the null hypothesis at the 5\% significance level?}
}

\pause

\includegraphics[width=\textwidth]{\chp7@path/7-4_power/figures/power/power_null_B_0_1-7_with_rejection_region}

\pause

The difference should be at least 
\[ 1.96 * 1.70 = 3.332 \] 
or at most 
\[ -1.96 * 1.70 = 3.332. \]

\end{frame}

%%%%%%%%%%%%%%%%%%%%%%%%%%%%%%%%%%%%

\begin{frame}
\frametitle{Example - BP, power}

{\dq
{\footnotesize
Suppose that the company researchers care about finding any effect on blood pressure that is 3 mmHg or larger vs the standard medication. What is the power of the test that can detect this effect?
}}

\pause

\includegraphics[width=\textwidth]{\chp7@path/7-4_power/figures/power/power_null_C_0_1-7_with_alt_at_3}

\pause

\[ Z = \frac{-3.332 - (-3)}{1.70} = -0.20 \]

\pause

\[ P(Z < -0.20) = 0.4207 \]

\end{frame}

%%%%%%%%%%%%%%%%%%%%%%%%%%%%%%%%%%%%

\section{Edfinity Quiz: Sample size for desired power}

%%%%%%%%%%%%%%%%%%%%%%%%%%%%%%%%%%%%

\begin{frame}
\frametitle{Example - BP, required sample size for 80\% power}

{\dq
{\footnotesize
What sample size will lead to a power of 80\% for this test?
}}

\pause

\includegraphics[width=\textwidth]{\chp7@path/7-4_power/figures/power/power_null_0_0-76_with_alt_at_3_and_shaded}

\pause

\[ SE = \frac{3}{2.8} = 1.07142 \]

\pause

\[ 1.07142 = \sqrt{ \frac{12^2}{n} + \frac{12^2}{n} } \]

\pause

\[ n = 250.88 \rightarrow n \ge 251 \]

\end{frame}

%%%%%%%%%%%%%%%%%%%%%%%%%%%%%%%%%%%%

\begin{frame}
\frametitle{Recap}

\begin{itemize}
\item Calculate required sample size for a desired level of power
\item Calculate power for a range of sample sizes, then choose the sample size that yields the target power (usually 80\% or 90\%)
\end{itemize}
\begin{center}
\includegraphics[width=0.7\textwidth]{\chp7@path/7-4_power/figures/power/power_curve_neg-3}
\end{center}

\end{frame}

%%%%%%%%%%%%%%%%%%%%%%%%%%%%%%%%%%%%

\section{Edfinity Quiz: How would you increase power in your experiments or study design?}  % Ask what students think before revealing the answers below; can be an open-answer question to get all their ideas

%%%%%%%%%%%%%%%%%%%%%%%%%%%%%%%%%%%%

\begin{frame}
\frametitle{Achieving desired power}

There are several ways to increase power (and hence decrease type 2 error rate):

\pause

{\small
\begin{enumerate}

\item Increase the sample size.

\pause

\item Decrease the standard deviation of the sample, which essentially has the same effect as increasing the sample size (it will decrease the standard error). With a smaller $s$ we have a better chance of distinguishing the null value from the observed point estimate. This is difficult to ensure but cautious measurement process and limiting the population so that it is more homogenous may help.

\pause

\item Increase $\alpha$, which will make it more likely to reject $H_0$ (but note that this has the side effect of increasing the Type 1 error rate).

\pause

\item Consider a larger effect size. If the true mean of the population is in the alternative hypothesis but close to the null value, it will be harder to detect a difference.

\end{enumerate}
}

\end{frame}

%%%%%%%%%%%%%%%%%%%%%%%%%%%%%%%%%%%%

%%%%%%%%%%%%%%%%%%%%%%%%%%%%%%%%%%%%
% End document
%%%%%%%%%%%%%%%%%%%%%%%%%%%%%%%%%%%%

\end{document}