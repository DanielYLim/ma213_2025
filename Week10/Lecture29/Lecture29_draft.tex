%%%%%%%%%%%%%%%%%%%%%%%%%%%%%%%%%%%%
% Slide options
%%%%%%%%%%%%%%%%%%%%%%%%%%%%%%%%%%%%

% Option 1: Slides with solutions

\documentclass[slidestop,compress,mathserif]{beamer}
\newcommand{\soln}[1]{\textit{#1}}
\newcommand{\solnGr}[1]{#1}

% Option 2: Handouts without solutions

%\documentclass[11pt,containsverbatim,handout]{beamer}
%\usepackage{pgfpages}
%\pgfpagesuselayout{4 on 1}[letterpaper,landscape,border shrink=5mm]
%\newcommand{\soln}[1]{ }
%\newcommand{\solnGr}{ }

%%%%%%%%%%%%%%%%%%%%%%%%%%%%%%%%%%%%
% Style
%%%%%%%%%%%%%%%%%%%%%%%%%%%%%%%%%%%%
\def\chp7@path{../../Chp 7}
\input{../../lec_style.tex}


%%%%%%%%%%%%%%%%%%%%%%%%%%%%%%%%%%%%
% Preamble
%%%%%%%%%%%%%%%%%%%%%%%%%%%%%%%%%%%%

\title[Lecture 29]{MA213: Lecture 29}
\subtitle{Module 4: Inference}
\author{OpenIntro Statistics, 4th Edition}
\institute{$\:$ \\ {\footnotesize Based on slides developed by Mine \c{C}etinkaya-Rundel of OpenIntro. \\
The slides may be copied, edited, and/or shared via the \webLink{http://creativecommons.org/licenses/by-sa/3.0/us/}{CC BY-SA license.} \\
Some images may be included under fair use guidelines (educational purposes).}}
\date{}


%%%%%%%%%%%%%%%%%%%%%%%%%%%%%%%%%%%%
% Begin document
%%%%%%%%%%%%%%%%%%%%%%%%%%%%%%%%%%%%

\begin{document}


%%%%%%%%%%%%%%%%%%%%%%%%%%%%%%%%%%%%
% Title page
%%%%%%%%%%%%%%%%%%%%%%%%%%%%%%%%%%%%

{
\addtocounter{framenumber}{-1} 
{\removepagenumbers 
\usebackgroundtemplate{\includegraphics[width=\paperwidth]{../../OpenIntro_Grid_4_3-01.jpg}}
\begin{frame}

\hfill \includegraphics[width=20mm]{../../oiLogo_highres}

\titlepage

\end{frame}
}
}


%%%%%%%%%%%%%%%%%%%%%%%%%%%%%%%%%%%%
% Recap/Agenda 
%%%%%%%%%%%%%%%%%%%%%%%%%%%%%%%%%%%%
% TODO better formatting
\begin{frame}
    \frametitle{Module 4: Inference}
    \begin{itemize}
        \item \hl{Previously: }Power calculations for a difference of means (Chapter 7.4)
        \item \hl{This time: }Comparing many means with ANOVA (Chapter 7.5)
        \item \hl{Reading: }Chapter 8.1 for next time
        \item \hl{Deadlines/Announcements: }HW 4.3 due Monday
    \end{itemize}
    
\end{frame}
%%%%%%%%%%%%%%%%%%%%%%%%%%%%%%%%%%%%
% Sections
%%%%%%%%%%%%%%%%%%%%%%%%%%%%%%%%%%%%

%%%%%%%%%%%%%%%%%%%%%%%%%%%%%%%%%%%

\section{Review}

%%%%%%%%%%%%%%%%%%%%%%%%%%%%%%%%%%%

\begin{frame}
    \frametitle{Review: Z- and t-tests}

\end{frame}

\begin{frame}
    \frametitle{Review, continued}

    ANOVA allows us to compare 3+ means in different levels/categories, for example in (motivating lecture example).

    F-statistic...
\end{frame}

%%%%%%%%%%%%%%%%%%%%%%%%%%%%%%%%%%%

\section{ANOVA, continued}

%%%%%%%%%%%%%%%%%%%%%%%%%%%%%%%%%%%

\begin{frame}
\frametitle{p-value in ANOVA}

\vspace{-0.25cm}

{\footnotesize
\begin{center}
\begin{tabular}{ll rrr>{\columncolor[gray]{.6}[.5\tabcolsep]}rr}
\hline
                & 			& Df 	& Sum Sq	& Mean Sq 	& F value 	& Pr($>$F) \\ 
\hline
(\hl{G}roup) 	& depth 		& 2 	& 16.96	& 8.48 		& \orange{6.14} 	& 0.0063 \\ 
(\hl{E}rror) 	& Residuals 	& 27 	& 37.33 	& 1.38 		&  		&  \\ 
\hline
                & \hl{T}otal	& 29	& 54.29 \\
\end{tabular}
\end{center}
}

\formula{p-value}
{
p-value is the probability of at least as large a ratio between the ``between group" and ``within group" variability, if in fact the means of all groups are equal. It's calculated as the area under the F curve, with degrees of freedom $df_G$ and $df_E$, above the observed F statistic.
}

\pause

\includegraphics[width=0.9\textwidth]{\chp7@path/7-5_anova/figures/aldrin/f}

\end{frame}

%%%%%%%%%%%%%%%%%%%%%%%%%%%%%%%%%%%

\section{Edfinity Quiz: ANOVA practice}

%%%%%%%%%%%%%%%%%%%%%%%%%%%%%%%%%%%

\section{R Demonstration: ANOVA, interpretation}

%%%%%%%%%%%%%%%%%%%%%%%%%%%%%%%%%%%

\begin{frame}
\frametitle{Conclusion - in context}

\pq{What is the conclusion of the hypothesis test?}

$\:$ \\

The data provide convincing evidence that the average aldrin concentration
\begin{enumerate}[(a)]

\item  is different for all groups.

\item on the surface is lower than the other levels.

\solnMult{is different for at least one group.}

\item is the same for all groups.

\end{enumerate}

\end{frame}

%%%%%%%%%%%%%%%%%%%%%%%%%%%%%%%%%%%

\begin{frame}
\frametitle{Conclusion}

\begin{itemize}

\item  If p-value is small (less than $\alpha$), reject $H_0$. The data provide convincing evidence that at least one mean is different from (but we can't tell which one).

\pause

\item If p-value is large, fail to reject $H_0$. The data do not provide convincing evidence that at least one pair of means are different from each other, the observed differences in sample means are attributable to sampling variability (or chance).

\end{itemize}

\end{frame}

%%%%%%%%%%%%%%%%%%%%%%%%%%%%%%%%%%%

\subsection{Checking conditions}

%%%%%%%%%%%%%%%%%%%%%%%%%%%%%%%%%%%

\begin{frame}[fragile]
\frametitle{(1) independence}

\dq{Does this condition appear to be satisfied?}

\soln{\only<2>{In this study the we have no reason to believe that the aldrin concentration won't be independent of each other.}}

\end{frame}

%%%%%%%%%%%%%%%%%%%%%%%%%%%%%%%%%%%

\begin{frame}[fragile]
\frametitle{(2) approximately normal}

\dq{Does this condition appear to be satisfied?}

\begin{center}
\includegraphics[width=\textwidth]{\chp7@path/7-5_anova/figures/aldrin/normal}
\end{center}

\end{frame}

%%%%%%%%%%%%%%%%%%%%%%%%%%%%%%%%%%%

\begin{frame}[fragile]
\frametitle{(3) constant variance}

\dq{Does this condition appear to be satisfied?}

\begin{center}
\includegraphics[width=0.7\textwidth]{\chp7@path/7-5_anova/figures/aldrin/homo}
\end{center}

\end{frame}

%%%%%%%%%%%%%%%%%%%%%%%%%%%%%%%%%%%

\subsection{Multiple comparisons \& Type 1 error rate}

%%%%%%%%%%%%%%%%%%%%%%%%%%%%%%%%%%%

\begin{frame}
\frametitle{Which means differ?}

\begin{itemize}

\item Earlier we concluded that at least one pair of means differ. The natural question that follows is ``which ones?"

\pause

\item We can do two sample $t$ tests for differences in each possible pair of groups.

\pause

\end{itemize}

\dq{Can you see any pitfalls with this approach?}

\pause

\begin{itemize}

\item When we run too many tests, the Type 1 Error rate increases.

\item This issue is resolved by using a modified significance level.

\end{itemize}

\end{frame}

%%%%%%%%%%%%%%%%%%%%%%%%%%%%%%%%%%%

\begin{frame}
\frametitle{Multiple comparisons}

\begin{itemize}

\item The scenario of testing many pairs of groups is called \hl{multiple comparisons}.

\pause

\item The \hl{Bonferroni correction} suggests that a more \orange{stringent} significance level is more appropriate for these tests:

\[ \alpha^\star = \alpha / K \]

where $K$ is the number of comparisons being considered.

\pause

\item If there are $k$ groups, then usually all possible pairs are compared and $K = \frac{k (k - 1)}{2}$.

\end{itemize}

\end{frame}

%%%%%%%%%%%%%%%%%%%%%%%%%%%%%%%%%%%

\begin{frame}
\frametitle{Determining the modified $\alpha$}

\pq{In the aldrin data set depth has 3 levels: bottom, mid-depth, and surface. If $\alpha = 0.05$, what should be the modified significance level for two sample $t$ tests for determining which pairs of groups have significantly different means?}

\begin{enumerate}[(a)]
\item $\alpha^* = 0.05$
\item $\alpha^* = 0.05 / 2 = 0.025$
\solnMult{$\alpha^* = 0.05 / 3 = 0.0167$}
\item $\alpha^* = 0.05 / 6 = 0.0083$
\end{enumerate}

\end{frame}

%%%%%%%%%%%%%%%%%%%%%%%%%%%%%%%%%%%

\begin{frame}
\frametitle{Which means differ?}

\pq{Based on the box plots below, which means would you expect to be significantly different?}

\twocol{0.6}{0.4}{
\includegraphics[width=\textwidth]{\chp7@path/7-5_anova/figures/aldrin/homo}
}
{
\begin{enumerate}[(a)]
\item bottom \& surface
\item bottom \& mid-depth
\item mid-depth \& surface
\item bottom \& mid-depth; mid-depth \& surface
\item bottom \& mid-depth; bottom \& surface; mid-depth \& surface
\end{enumerate}
}

\end{frame}

%%%%%%%%%%%%%%%%%%%%%%%%%%%%%%%%%%

\begin{frame}
\frametitle{Which means differ? (cont.)}

If the ANOVA assumption of equal variability across groups is satisfied, we can use the data from all groups to estimate variability:
    
\begin{itemize}

\item Estimate any within-group standard deviation with $\sqrt{MSE}$, which is $s_{pooled}$

\item Use the error degrees of freedom, $n - k$, for $t$-distributions

\end{itemize}

\formula{Difference in two means: after ANOVA}{
\[ SE = \sqrt{  \frac{\sigma_1^2}{n_1} + \frac{\sigma_2^2}{n_2} } \approx \sqrt{ \frac{MSE}{n_1} + \frac{MSE}{n_2} } \]
}

\end{frame}

%%%%%%%%%%%%%%%%%%%%%%%%%%%%%%%%%%

\begin{frame}
\frametitle{}

\dq{Is there a difference between the  average aldrin concentration at the bottom and at mid depth?}

\twocol{0.4}{0.7}{
{\scriptsize
\begin{center}
\begin{tabular}{l | c c c}
        & n	& mean	& sd		\\
\hline
bottom	& 10	& \orange{6.04}	& 1.58 \\
middepth& 10	& \orange{5.05}	& 1.10 \\
surface	& 10	& 4.2 	& 0.66 \\
\hline
overall	& 30	& 5.1		& 1.37
\end{tabular}
\end{center}
}
}
{
{\scriptsize
\begin{center}
\begin{tabular}{l rrrrr}
\hline
                & Df 	& Sum Sq	& Mean Sq 	& F value 	& Pr($>$F) \\ 
\hline
depth 		& 2 	& 16.96 	& 8.48 		& 6.13 	& 0.0063 \\ 
Residuals 	& \orange{27} 	& 37.33 	& \orange{1.38} 		&  		&  \\ 
\hline
Total			& 29	& 54.29 \\
\end{tabular}
\end{center}
}
}

\begin{eqnarray*}
T_{df_E} &=& \frac{(\bar{x}_{bottom} - \bar{x}_{middepth})}{\sqrt{ \frac{MSE}{n_{bottom}} + \frac{MSE}{n_{middepth}} }} \\ 
\pause
T_{27} &=& \frac{( 6.04 - 5.05 )}{\sqrt{ \frac{1.38}{10} + \frac{1.38}{10} }} = \frac{0.99}{0.53}  =1.87 \\
\pause
0.05 &<& p-value < 0.10 \qquad \text{{\footnotesize (two-sided)}} \\
\pause
\alpha^\star &=& 0.05 / 3 = 0.0167
\end{eqnarray*}

\pause
{\small Fail to reject $H_0$, data do not provide convincing evidence of a difference between average aldrin concentrations at bottom and mid depth.}

\end{frame}

%%%%%%%%%%%%%%%%%%%%%%%%%%%%%%%%%%

\begin{frame}
\frametitle{}

\app{Pairwise comparisons}{Is there a difference between the  average aldrin concentration at the bottom and at surface?}

\pause

\soln{
\begin{eqnarray*}
T_{df_E} &=& \frac{(\bar{x}_{bottom} - \bar{x}_{surface})}{\sqrt{ \frac{MSE}{n_{bottom}} + \frac{MSE}{n_{surface}} }} \\ 
\pause
T_{27} &=& \frac{( 6.04 - 4.2 )}{\sqrt{ \frac{1.38}{10} + \frac{1.38}{10} }} = \frac{1.84}{0.53}  =3.47 \\
\pause
p-value = 0.0018 \qquad \text{{\footnotesize (two-sided)}} \\
\pause
\alpha^\star &=& 0.05 / 3 = 0.0167
\end{eqnarray*}
\pause
{\small Reject $H_0$, the data provide convincing evidence of a difference between the average aldrin concentrations at bottom and surface.}
}

\end{frame}

%%%%%%%%%%%%%%%%%%%%%%%%%%%%%%%%%%

%%%%%%%%%%%%%%%%%%%%%%%%%%%%%%%%%%    


%%%%%%%%%%%%%%%%%%%%%%%%%%%%%%%%%%%%
% End document
%%%%%%%%%%%%%%%%%%%%%%%%%%%%%%%%%%%%

\end{document}