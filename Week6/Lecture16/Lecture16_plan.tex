

%%%%%%%%%%%%%%%%%%%%%%%%%%%%%%%%%%%%
% Lesson Plan (50 minutes)
%%%%%%%%%%%%%%%%%%%%%%%%%%%%%%%%%%%%
\begin{frame}
\frametitle{Lesson Plan}
\begin{itemize}
    \item xx min Lecture: Reminder of sampling distribution
    \item xx min R demonstration: motivate CLT
    \item xx min Board work: Given that the Binomial converges to Normal, derive mean and std error
    \item xx min R demonstration: superimpose Normal on simulated sampling dist, show approximation for varying N
    \item xx min Lecture: CLT
    \item xx min Edfinity quiz: Checking sample size
    \item xx min lecture: Wrap up CLT
 \end{itemize}
\end{frame}
    
%%%%%%%%%%%%%%%%%%%%%%%%%%%%%%%%%%%%
% Learning objectives:
%%%%%%%%%%%%%%%%%%%%%%%%%%%%%%%%%%%%
\begin{frame}
\frametitle{Learning Objectives}
\begin{itemize}
    \item \textbf{M1 LO3: Use R for Data Management and Exploration:} Utilize R to load, pre-process, and explore data through visualization and summarization techniques.
    \item \textbf{M3 LO2: Visualize and Interpret Sampling Distributions:} Draw and interpret sampling distributions for a point estimate (e.g., population proportion) across different sample sizes, explaining how the distribution changes as the sample size increases. [Q3, L4]     \item \textbf{M6 LO1: Validate and Explain Probability Distributions:} Assess the validity of a probability distribution using the concepts of outcome, sample space, and probability properties (e.g., disjoint outcomes, probabilities between 0 and 1, and total probabilities summing to 1).
    \item \textbf{M3 LO3: Calculate and Interpret Standard Error:} Calculate the standard error for proportions and interpret it as a measure of sampling variability. [Q3, L4]
\end{itemize}
\end{frame}

%%%%%%%%%%%%%%%%%%%%%%%%%%%%%%%%%%%%
% TODO: Copy and adapt these slides base on the lesson plan
\input{\chp5@path/5-1_point_est_sampling_var/5-1_point_est_sampling_var.tex}
