

%%%%%%%%%%%%%%%%%%%%%%%%%%%%%%%%%%%%
% Lesson Plan (50 minutes)
%%%%%%%%%%%%%%%%%%%%%%%%%%%%%%%%%%%%
\begin{frame}
\frametitle{Lesson Plan}
\begin{itemize}
    \item xx min Lecture: Motivate CIs
    \item xx min Board work: given N(mu, sterr), derivation of interval with target level
    \item xx min Lecture: Facebook example
    \item xx min R demonstration: simulate running the experiment many times, get different answers for the CI, 95\% contain true
    \item xx min Edfinity quiz (confidence level)
    \item xx min R demonstration: effect of confidence level
    \item xx min lecture: Wrap up confidence intervals
\end{itemize}
\end{frame}
        
%%%%%%%%%%%%%%%%%%%%%%%%%%%%%%%%%%%%
% Learning objectives:
%%%%%%%%%%%%%%%%%%%%%%%%%%%%%%%%%%%%
\begin{frame}
\frametitle{Learning Objectives}
\begin{itemize}
    \item \textbf{M1 LO3: Use R for Data Management and Exploration:} Utilize R to load, pre-process, and explore data through visualization and summarization techniques.
    \item \textbf{M3 LO3: Calculate and Interpret Standard Error:} Calculate the standard error for proportions and interpret it as a measure of sampling variability. 
    \item \textbf{M3 LO4: Explain Hypothesis Testing and Its Limitations:} Discuss the use cases and potential issues with hypothesis testing, including the interpretation of results. 
    \item \textbf{M4 LO1: Design and Interpret Confidence Intervals:} Design, execute, and interpret confidence intervals for the population proportion. 
\end{itemize}
\end{frame}

%%%%%%%%%%%%%%%%%%%%%%%%%%%%%%%%%%%%
% TODO: Make lecture draft file
% TODO: Copy in slides from section 5.2
% TODO: prep board work on normal distribution CI
% TODO: R Demo on different CIs every time, coverage
% TODO: Edfinity quiz on confidence levels
% TODO: R demo on effect of confidence level

